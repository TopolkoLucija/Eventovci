\chapter{Implementacija i korisničko sučelje}
		
		
		\section{Korištene tehnologije i alati}
					
					
			Članovi tima komunicirali su primarno putem aplikacije Whatsapp\footnote{https://www.whatsapp.com/}. WhatsApp je mobilna aplikacija za besplatnu razmjenu poruka, fotografija, videozapisa i drugih datoteka, te uspostavljanje glasovnih i videopoziva putem interneta. Komunikacija s asistenticom i demonstratorom ostvarena je aplikacijama Microsoft Teams\footnote{https://www.microsoft.com/en-us/microsoft-teams/group-chat-software} te Microsoft Outlook\footnote{https://www.microsoft.com/en-us/microsoft-365/outlook/}. Microsoft Teams je platforma za poslovnu komunikaciju, a Microsoft Outlook program za primanje i slanje elektroničke pošte. Oba proizvoda razvila je tvrtka Microsoft. Sastanci na daljinu također su održani putem aplikacije Microsoft Teams. Raspodjela zadataka i praćenje napretka ostvarena je pomoću Notiona\footnote{https://www.notion.so/}. Notion je aplikacija koja nudi organizacijske alate uključujući upravljanje zadacima, praćenje projekta, popise obaveza i označavanje.
			
			Dijagrami su izrađeni pomoću aplikacije Astah UML\footnote{https://astah.net/}, alata za UML modeliranje. Za izradu modela baze podataka korišten je alat ERDPlus\footnote{https://erdplus.com/}.
			
			Sustav korišten za upravljanje izvornim kodom je Git\footnote{https://git-scm.com/}, a udaljeni repozitorij projekta dostupan je na web platformi GitHub\footnote{https://github.com/}. GitHub služi za razvoj softvera i kontrolu verzija koja programerima omogućuje stvaranje, pohranu i upravljanje svojim kodom.
			
			Backend je pisan u programskom jeziku Java\footnote{https://www.java.com/}, pomoću radnog okvira Spring Boot\footnote{https://spring.io/}. Java je popularan objektno orijentiran programski jezik, a Spring Boot rješenje za stvaranje Spring aplikacija produkcijske razine s minimalnom količinom konfiguracije. Kao razvojno okruženje korišten je Intellij IDEA Ultimate\footnote{https://www.jetbrains.com/idea/download/}.
			
			Za razvoj frontenda korišteni su programski jezik JavaScript\footnote{https://www.javascript.com/} i biblioteka React\footnote{https://react.dev/}. Javascript je skriptni programski jezik koji se izvršava u web pregledniku na strani korisnika. React je njegova biblioteka otvorenog koda za izgradnju korisničkih sučelja temeljenih na komponentama. Razvojno okruženje je Visual Studio Code\footnote{https://code.visualstudio.com/}. 
			
			Lokalno pokrenuta web aplikacija koristi sustav za upravljanje bazama podataka H2\footnote{https://www.h2database.com/html/main.html}, dok aplikacija pogonjena pomoću platforme Render\footnote{https://render.com/} koristi sustav za upravljanje bazama PostgreSQL\footnote{https://www.postgresql.org/}. Razlika između H2 i PostgreSQL-a je u tome što H2 služi za testiranje i podaci se gube nakon gašenja sustava, a PostgreSQL trajno čuva podatke. 
			
			Za automatsko ispitivanje komponenti korišten je programski jezik Java i okvir JUnit 5\footnote{https://junit.org/junit5/}, a za automatsko ispitivanje sustava alat Selenium IDE\footnote{https://www.selenium.dev/selenium-ide/}. Selenium IDE omogućava snimanje, uređivanje i reproduciranje testova direktno u web pregledniku bez potrebe za pisanjem koda.
			
			Dokumentacija je pisana u programskom jeziku LaTeX\footnote{https://www.latex-project.org/}, u TeXstudiu\footnote{https://www.texstudio.org/}. TeXStudio je uređivač otvorenog koda namijenjen za LaTeX, programski jezik za pisanje strukturiranih tekstova i njihov automatski slog i prijelom u dokumente profesionalne kvalitete spremne za tisak.
						 
			
			\eject 
		
	
		\section{Ispitivanje programskog rješenja}
			
			\textbf{\textit{dio 2. revizije}}\\
			
			 \textit{U ovom poglavlju je potrebno opisati provedbu ispitivanja implementiranih funkcionalnosti na razini komponenti i na razini cijelog sustava s prikazom odabranih ispitnih slučajeva. Studenti trebaju ispitati temeljnu funkcionalnost i rubne uvjete.}
	
			
			\subsection{Ispitivanje komponenti}
			\textit{Potrebno je provesti ispitivanje jedinica (engl. unit testing) nad razredima koji implementiraju temeljne funkcionalnosti. Razraditi \textbf{minimalno 6 ispitnih slučajeva} u kojima će se ispitati redovni slučajevi, rubni uvjeti te izazivanje pogreške (engl. exception throwing). Poželjno je stvoriti i ispitni slučaj koji koristi funkcionalnosti koje nisu implementirane. Potrebno je priložiti izvorni kôd svih ispitnih slučajeva te prikaz rezultata izvođenja ispita u razvojnom okruženju (prolaz/pad ispita). }
			
			
			
			\subsection{Ispitivanje sustava}
			
			 \textit{Potrebno je provesti i opisati ispitivanje sustava koristeći radni okvir Selenium\footnote{\url{https://www.seleniumhq.org/}}. Razraditi \textbf{minimalno 4 ispitna slučaja} u kojima će se ispitati redovni slučajevi, rubni uvjeti te poziv funkcionalnosti koja nije implementirana/izaziva pogrešku kako bi se vidjelo na koji način sustav reagira kada nešto nije u potpunosti ostvareno. Ispitni slučaj se treba sastojati od ulaza (npr. korisničko ime i lozinka), očekivanog izlaza ili rezultata, koraka ispitivanja i dobivenog izlaza ili rezultata.\\ }
			 
			 \textit{Izradu ispitnih slučajeva pomoću radnog okvira Selenium moguće je provesti pomoću jednog od sljedeća dva alata:}
			 \begin{itemize}
			 	\item \textit{dodatak za preglednik \textbf{Selenium IDE} - snimanje korisnikovih akcija radi automatskog ponavljanja ispita	}
			 	\item \textit{\textbf{Selenium WebDriver} - podrška za pisanje ispita u jezicima Java, C\#, PHP koristeći posebno programsko sučelje.}
			 \end{itemize}
		 	\textit{Detalji o korištenju alata Selenium bit će prikazani na posebnom predavanju tijekom semestra.}
			
			\eject 
		
		
		\section{Dijagram razmještaja}
			
			\textbf{\textit{dio 2. revizije}}
			
			 \textit{Potrebno je umetnuti \textbf{specifikacijski} dijagram razmještaja i opisati ga. Moguće je umjesto specifikacijskog dijagrama razmještaja umetnuti dijagram razmještaja instanci, pod uvjetom da taj dijagram bolje opisuje neki važniji dio sustava.}
			
			\eject 
		
		\section{Upute za puštanje u pogon}
		
			
			\textit{Za deploy aplikacije korištena je platforma Render. Putem njega omogućeno je da aplikacija bude javno dostupna. Pri implementaciji web usluge na render najprije je s njime potrebno povezati svoj GitHub repozitorij, pripremiti projekt za deploy te kreirati potrebne dijelove na Render-u.}
			
			\subsection{Priprema projekta za deploy}
			\textit{Priprema projekta za deploy na Render uključuje pripremu backenda i pripremu frontenda. Za pripremu backenda potrebno je dodati varijable okruženja, dodati DockerFile te postaviti /api kao prefiks svim zahtjevima na backend, dok je za pripremu frontenda potrebno dodati dependencye u package.json, dodati /src/setupProxy.js koji služi kao proxy server za lokalni development te dodati app.js u kojem se nalazi express server za posluživanje frontenda. Zatim je potrebno u Rander kontolnoj ploči kreirati bazu podataka, backend i fronted.}
			
			\subsection{Kreiranje potrebnih dijelova na Render-u}
			\textit{Prvo kreiramo bazu podataka tako što odaberemo opciju Novo -> PostgreSQL u Renderovoj kontrolnoj ploči. Postavimo ime baze i korisničko ime te kliknemo Stvori Bazu podataka. Zatim kreiramo backend tako što odaberemo Novo -> Web Service. Ukoliko nismo povezali račun sada moramo kako bi mogli odabrati projekt kojeg želimo deployati. Nakon toga sve što je potrebno je postaviti ime za servis koje postaje dio web adrese, postavitit korijennski direktorij, dodati potrebne varijable okruženja (tj. kopirati njihove vrijednosti iz postavki baze podataka na Renderu), za environment odabrati Docker te postaviti putanju za Dockerfile. Tada odaberemo stvoriti Web uslugu. Posljednji korak je kreiranje frontenda pomoću odabira Novo -> Web Service. Zatim se odabire projekt, ime za servis, postavi korijenski direktorij, za environment odabire Node te za region Frankfurt, postavljamo komande upravljanja i dodamo potrebne varijable okruženja. Tada odaberemo stvoriti Web Service.
			
			
			\eject 